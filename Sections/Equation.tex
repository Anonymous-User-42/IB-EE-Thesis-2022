
\section{{Equations of Motion}}
            
    {We know that,}
            
        $$\frac{d}{dt}\left(\frac{\partial L}{\partial \dot{q}_j}\right) - \frac{\partial L}{\partial q_j} + \frac{\partial R}{\partial q_j} = 0$$
            
    {Where L is the \textbf{Lagrangian}, $q_i$ is the \textbf{generalized coordinates}, \textit{R} is the \textbf{Rayleigh dissipation function} and "$Q_j$" is the \textbf{generalized forces}.}
            
    {According to our specific case that we are investigating, we have two generalized coordinates, "$\theta$" and "\textit{x}".}        
            
    {Therefore the \textbf{coupled set of differential equations} that describe the motion of the damped elastic harmonic pendulum (Equations of Motion) with respect to the two generalized coordinates x and $\theta$ are,}
            
        $$m\cdot\ddot{x} -m\cdot(l_0+x)\dot{\theta}^2 + kx - g\cdot m\cdot\cos{\theta} + C_D\dot{x} = 0$$
            
        $$m\cdot(l_0+x)^2\ddot{\theta} + 2m\cdot(l_0+x)\dot{x}\dot{\theta} + g\cdot m\cdot(l_0+x)\sin{\theta} + C_D\dot{\theta} = 0$$
            
    {With $\ddot{x}$ and $\ddot{\theta}$ isolated the equations are,}
            
        $$\ddot{x} = (l_0+x)\dot{\theta}^2 - \frac{k}{m}\cdot x + g\cdot\cos{\theta} - \frac{C_D}{m}\dot{x}$$
            
        $$\ddot{\theta} = - \frac{2}{(l_0+x)}\dot{x}\dot{\theta} - \frac{g}{(l_0+x)}\cdot\sin{\theta} - \frac{C_D}{m\cdot(l_0+x)^2}\dot{\theta}$$
            
    {Where, \textit{m} is the \textbf{mass}, $l_0$ is the \textbf{rest length} of the spring, \textit{x} is the \textbf{extension length} of the spring, $\theta$ is the \textbf{angle made between the spring and the normal}, \textit{g} is the \textbf{gravitational acceleration}, \textit{k} is the \textbf{spring stiffness constant} and $C_D$ \textbf{damping coefficient}.}
            
    {\textbf{Note}: When damping is zero (Absolute Harmonic Motion), $C_D$ is zero.}
            
    {These equations will be directly inputted into the python source code from \textbf{appendix \ref{pycode}} to solve these equations \textbf{numerically} with an accuracy of $\pm 10^{-8}$ for each time step of the \textbf{Runge-Kutta 4th Order} (RK4) algorithm.}
           
	{\textbf{Note}: The numerical value of each time step involved in the simulation is $10^{-4}$ of a second.}
           
\section{{Drag Equation}}
            
    {We know that the general Drag equation is mathematically defined as,}
            
        $$F_D = \frac{1}{2}\rho v^2C_DA$$
           
    {Where $F_D$ is the \textbf{Air/Fluid resistance} between the mass and the fluid, $\rho$ is the \textbf{density of the fluid}, \textit{v} is the \textbf{speed of the object} relative to the fluid, $C_D$ is \textbf{velocity decay constant} (damping constant) and \textit{A} is the \textbf{cross sectional area}.}
            
    {But as the fluid medium that we are dealing with is simply \textbf{atmospheric air} at $25^\circ$ and at 1 atm pressure, $\rho = 1.1839 kg/m^3$. Also that, the physical nature of the mass that we are dealing with is \textbf{spherical}, therefore $C_D = 0.47$ and $A = \pi r^2$.}
            
    {Therefore, we have,}
            
        $$F_D = \pi\cdot\frac{432165}{10^{6}}\cdot r^2v^2$$
            
\section{{Stokes Law Equation}}
            
    {We know that,}
            
        $$F_D = 6\pi\mu Rv_D$$
            
    {Where $F_D$ is the \textbf{Air/Fluid resistance} between the mass and the fluid, $\mu$ is the \textbf{dynamic viscosity} of the fluid, \textit{R} is the \textbf{radius} of the mass and $v_D$ is the \textbf{flow velocity} relative to the mass.}
            
    {But as the fluid medium that we are dealing with is simply \textbf{atmospheric air} at $25^\circ$ and at 1 atm pressure, $\mu = 18.6 \mu P\cdot s = 1.86 \times 10^{-5} P\cdot s$.}
            
    {Also with close observation it can be inferred that the magnitude of \textbf{flow velocity} relative to the mass is the same as the magnitude of the \textbf{absolute velocity} of the system. Therefore using \textbf{Newtonian vector} definition, we define,}
            
        $$v = \sqrt{(\ddot{x})^2 + (\ddot{\theta})^2}$$
        
    {Therefore we can infer that,}
        
        $$v_D = \sqrt{(\ddot{x})^2 + (\ddot{\theta})^2}$$
            
    {Therefore, we finally have,}
            
        $$F_D = \frac{11.16}{10^{5}}\cdot\pi\sqrt{(\ddot{x})^2 + (\ddot{\theta})^2}\cdot R$$
            
\section{{Frequency Equations}}
            
    {The frequency of the system can be broken down into two types of frequencies if we consider it from the perspective of the two generalized coordinates. They are radial and angular frequencies}
            
    \subsection{{Radial frequency}}
                
        {Radial frequency is defined as the number of oscillations of the radial movement (Spring component of the system) per second. Physically and mathematically defined as,}
                
            $$\omega_r = \sqrt{\frac{k}{m}}$$
                
        {With observation, it can be inferred that, radial frequency remains constant whatsoever in the whole system for a particular case when \textit{\textbf{m}} is a constant.}
            
    \subsection{{Angular frequency}}
                
        {Angular frequency is defined as the number of oscillations of the angular movement (Pendulum component of the system) per second. Physically and mathematically defined as,}
                
            $$\omega_\theta = \sqrt{\frac{g}{l_0 + x}}$$
                
        {With observation, it can be inferred that, angular frequency is a function of radial position and will be a variable all throughout in the whole system.}
                
    \subsection{{Absolute frequency}}
                
        {If we are to talk about absolute frequency, we can define absolute frequency using Newtonian vector definition as,}
                
            $$f = \sqrt{\left(\omega_r\right)^2 + \left(\omega_\theta\right)^2}$$
                
        {Therefore we have,}
                
            $$f = \sqrt{\frac{k}{m} + \frac{g}{l_0 + x}}$$



\section{{Fundamental Derived Equations}}
{The fundamental derived equations we shall be using in this investigation are:}
            
    \begin{equation}
        \ddot{x} = (l_0+x)\dot{\theta}^2 - \frac{k}{m}\cdot x + g\cdot\cos{\theta} - \frac{C_D}{m}\dot{x}
        \label{eq1}
    \end{equation}
            
    \begin{equation}
        \ddot{\theta} = - \frac{2}{(l_0+x)}\dot{x}\dot{\theta} - \frac{g}{(l_0+x)}\cdot\sin{\theta} - \frac{C_D}{m\cdot(l_0+x)^2}\dot{\theta}
    \label{eq2}
    \end{equation}
            
        {\textbf{Note}: The coefficient of damping ($C_D$) from now on will be equal to the numeric value of 0.47.}
            
    \begin{equation}
        F_D = \pi\cdot\frac{432165}{10^{6}}\cdot r^2v^2
        \label{eq3}
    \end{equation}
            
    \begin{equation}
        F_D = \frac{11.16}{10^{5}}\cdot\pi\sqrt{(\ddot{x})^2 + (\ddot{\theta})^2}\cdot R
        \label{eq4}
    \end{equation}
            
        {\textbf{Note}: $R$ and $r$ represent the same physical quantity, i.e., the radius of the mass used in a specific case.}
            
    \begin{equation}
        \omega_r = \sqrt{\frac{k}{m}}
        \label{eq5}
    \end{equation}
            
    \begin{equation}
        \omega_\theta = \sqrt{\frac{g}{l_0 + x}}
        \label{eq6}
    \end{equation}
            
    \begin{equation}
        f = \sqrt{\frac{k}{m} + \frac{g}{l_0 + x}}
        \label{eq7}
    \end{equation}
            
{\textbf{Note}: All the equations mentioned above shall be inputted algorithmically in an computer program/software and shall be numerically estimated to about an accuracy of approximately $\pm10^{-8}$ of each time step interval.}
            
            




