\section{{Radial Displacement versus Time}}
        
    \begin{table}[H]
                \centering
                \begin{tabular}{|c|c|c|c|c|c|c|c|c|c|}
                \hline
                \hline
                \diagbox[width=5em]{\textit{Mass}}{\textit{Time}} & 0 & 2.5 & 5.0 & 7.5 & 10.0 & 12.5 & 15.0 & 17.5 & 20.0 \\
                \hline
                \hline
                1 & 0 & 0.008 & 0.086 & 0.107 & 0.090 & 0.096 & 0.097 & 0.097 & 0.098 \\
                \hline
                2 & 0 & 0.110 & 0.330 & 0.350 & 0.298 & 0.227 & 0.168 & 0.161 & 0.195 \\
                \hline
                3 & 0 & 0.558 & 0.420 & 0.021 & 0.258 & 0.537 & 0.522 & 0.440 & 0.301 \\
                \hline
                4 & 0 & 1.062 & -0.003 & 0.831 & 0.262 & 0.2031 & 0.805 & 0.332 & 0.157 \\
                \hline
                5 & 0 & 1.189 & 0.165 & 0.611 & 0.431 & 0.583 & 0.280 & 0.890 & 0.086 \\
                \hline
                6 & 0 & 1.079 & 0.924 & -0.012 & 1.313 & 0.228 & 0.759 & 0.596 & 0.396 \\
                \hline
                7 & 0 & 0.854 & 1.686 & 0.223 & 0.182 & 1.556 & 0.211 & 0.652 & 1.334 \\
                \hline
                8 & 0 & 0.579 & 1.832 & 1.230 & -0.007 & 0.493 & 1.746 & 0.471 & 0.162 \\
                \hline
                9 & 0 & 0.308 & 1.467 & 2.121 & 1.094 & -0.114 & 0.455 & 1.825 & 1.403 \\
                \hline
                10 & 0 & 0.062 & 0.923 & 2.062 & 2.204 & 1.304 & -0.023 & 0.144 & 1.196 \\
                \hline
                \hline
                \end{tabular}
                \caption{\textit{Experimental Data on \textbf{radial displacement} versus \textbf{time} from 0 to 20 seconds.}}
                \label{}
    \end{table}
        
	\textit{\textbf{Note}: Mass is in \textbf{kilograms}, Time is in \textbf{seconds}, and the values of radial displacement versus time is in \textbf{meters per second}.}        
        
\section{{Angular Displacement versus Time}}
        
        \begin{table}[H]
                \centering
                \begin{tabular}{|c|c|c|c|c|c|c|c|c|c|}
                \hline
                \hline
                \diagbox[width=5em]{\textit{Mass}}{\textit{Time}} & 0 & 2.5 & 5.0 & 7.5 & 10.0 & 12.5 & 15.0 & 17.5 & 20.0 \\
                \hline
                \hline
                1 & $\pi/2$ & 0.931 & 0.418 & -0.045 & -0.208 & -0.097 & 0.029 & 0.052 & 0.018 \\
                \hline
                2 & $\pi/2$ & 0.942 & 0.462 & 0.282 & 0.287 & 0.365 & 0.418 & 0.365 & 0.237 \\
                \hline
                3 & $\pi/2$ & 0.364 & -0.555 & -1.005 & -0.591 & -0.191 & 0.082 & 0.219 & 0.335 \\
                \hline
                4 & $\pi/2$ & 0.052 & -1.223 & -0.222 & 0.704 & 0.683 & 0.022 & -0.476 & -0.604 \\
                \hline
                5 & $\pi/2$ & -0.167 & -0.948 & 0.491 & 0.588 & -0.429 & -0.608 & 0.163 & 0.684 \\
                \hline
                6 & $\pi/2$ & -0.346 & -0.395 & 1.155 & -0.123 & -0.720 & 0.355 & 0.415 & -0.361 \\
                \hline
                7 & $\pi/2$ & -0.493 & -0.108 & 0.851 & -0.714 & 0.006 & 0.625 & -0.384 & -0.157 \\
                \hline
                8 & $\pi/2$ & -0.627 & 0.077 & 0.289 & -0.949 & 0.5312 & 0.007 & -0.404 & 0.526 \\
                \hline
                9 & $\pi/2$ & -0.754 & 0.250 & 0.020 & -0.395 & 0.808 & -0.5195 & 0.0485 & 0.166 \\
                \hline
                10 & $\pi/2$ & -0.879 & 0.464 & -0.134 & -0.062 & 0.290 & -0.601 & 0.591 & -0.115 \\
                \hline
                \hline
                \end{tabular}
                \caption{\textit{Experimental Data on \textbf{angular displacement} versus \textbf{time} from 0 to 20 seconds.}}
                \label{}
    \end{table}
    
	\textit{\textbf{Note}: Mass is in \textbf{kilograms}, Time is in \textbf{seconds}, and the values of angular displacement versus time is in \textbf{meters per second}.}
	        
\section{{Angular Frequency versus Time}}
        
        \begin{table}[H]
                \centering
                \begin{tabular}{|c|c|c|c|c|c|c|c|c|c|}
                \hline
                \hline
                \diagbox[width=5em]{\textit{Mass}}{\textit{Time}} & 0 & 2.5 & 5.0 & 7.5 & 10.0 & 12.5 & 15.0 & 17.5 & 20.0 \\
                \hline
                \hline
                1 & 3.130 & 3.118 & 3.006 & 2.976 & 2.994 & 2.991 & 2.989 & 2.989 & 2.988 \\
                \hline
                2 & 3.128 & 2.971 & 2.716 & 2.696 & 2.749 & 2.828 & 2.897 & 2.906 & 2.870 \\
                \hline
                3 & 3.129 & 2.509 & 2.630 & 3.099 & 2.794 & 2.526 & 2.539 & 2.607 & 2.771 \\
                \hline
                4 & 3.130 & 2.181 & 3.136 & 2.314 & 2.790 & 2.859 & 2.332 & 2.715 & 2.925 \\
                \hline
                5 & 3.128 & 2.117 & 2.906 & 2.471 & 2.627 & 2.492 & 2.768 & 2.274 & 2.996 \\
                \hline
                6 & 3.129 & 2.175 & 2.261 & 3.153 & 2.060 & 2.830 & 2.363 & 2.481 & 2.667 \\
                \hline
                7 & 3.133 & 2.303 & 1.912 & 2.834 & 2.883 & 1.960 & 2.848 & 2.538 & 2.086 \\
                \hline
                8 & 3.130 & 2.493 & 1.861 & 2.103 & 3.148 & 2.566 & 1.890 & 2.581 & 2.980 \\
                \hline
                9 & 3.131 & 2.741 & 1.995 & 1.773 & 2.168 & 3.331 & 2.597 & 1.865 & 2.046 \\
                \hline
                10 & 3.131 & 3.040 & 2.255 & 1.790 & 1.750 & 2.643 & 3.170 & 2.936 & 2.127 \\
                \hline
                \hline
                \end{tabular}
                \caption{\textit{Experimental Data on \textbf{angular frequency} versus \textbf{time} from 0 to 20 seconds.}}
                \label{}
    \end{table}
    
	\textit{\textbf{Note}: Mass is in \textbf{kilograms}, Time is in \textbf{seconds}, and the values of angular frequency versus time is in \textbf{Hertz}.}
        
\section{{Absolute Frequency versus Time}}
        
        \begin{table}[H]
                \centering
                \begin{tabular}{|c|c|c|c|c|c|c|c|c|c|}
                \hline
                \hline
                \diagbox[width=5em]{\textit{Mass}}{\textit{Time}} & 0 & 2.5 & 5.0 & 7.5 & 10.0 & 12.5 & 15.0 & 17.5 & 20.0 \\
                \hline
                \hline
                1 & 10.478 & 10.474 & 10.442 & 10.434 & 10.440 & 10.438 & 10.437 & 10.437 & 10.437 \\
                \hline
                2 & 7.7330 & 7.6700 & 7.5751 & 7.5677 & 7.5869 & 7.6157 & 7.6419 & 7.6452 & 7.6331 \\
                \hline
                3 & 6.5675 & 6.2949 & 6.3448 & 6.5533 & 6.4140 & 6.3022 & 6.3072 & 6.3353 & 6.4018 \\
                \hline
                4 & 5.8990 & 5.4550 & 5.9024 & 5.5098 & 5.7264 & 5.7581 & 5.5173 & 5.6895 & 5.7859 \\
                \hline
                5 & 5.4588 & 4.9477 & 5.3319 & 5.1085 & 5.1840 & 5.1188 & 5.2601 & 5.0174 & 5.3830 \\
                \hline
                6 & 5.1443 & 4.6250 & 4.6665 & 5.1564 & 4.5728 & 4.9661 & 4.7167 & 4.7776 & 4.8840 \\
                \hline
                7 & 4.9075 & 4.4259 & 4.2359 & 4.7248 & 4.7560 & 4.25726 & 4.7335 & 4.4975 & 4.3170 \\
                \hline
                8 & 4.7220 & 4.3255 & 3.9956 & 4.1124 & 4.7324 & 4.3660 & 4.0096 & 4.3801 & 4.6280 \\
                \hline
                9 & 4.5728 & 4.3177 & 3.8847 & 3.7756 & 3.9763 & 4.7117 & 4.2269 & 3.8194 & 3.9060 \\
                \hline
                10 & 4.4492 & 4.3860 & 3.8853 & 3.6340 & 3.6148 & 3.7780 & 4.476 & 4.3150 & 3.8050 \\
                \hline
                \hline
                \end{tabular}
                \caption{\textit{Experimental Data on \textbf{absolute frequency} versus \textbf{time} from 0 to 20 seconds.}}
                \label{}
    \end{table}
    
	\textit{\textbf{Note}: Mass is in \textbf{kilograms}, Time is in \textbf{seconds}, and the values of absolute frequency versus time is in \textbf{Hertz}.}
                




