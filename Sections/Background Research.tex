\textit{Before beginning this investigation we must first know some important facts, formulae, laws and be familiar with the concepts that are to be incorporated in this investigation.}

\section{\textit{The Lagrangian function}}
        
    \textit{Firstly, we must know that the fundamental concepts of Lagrangian mechanics and Lagrange's equations.}
        
    \textit{The Lagrangian is a function that is mathematically defined as,}
        
    $$L = T - V$$
        
    \textit{Where, L is the Lagrangian, T is the kinetic energy of the system and V is the potential energy of the system.}
        
    \textit{The Lagrange's equation of second kind or the Lagrange-Euler Equation is mathematically defined as,}
        
    $$\frac{d}{dt}\left(\frac{\partial L}{\partial \dot{q}_j}\right) - \frac{\partial L}{\partial q_j} = Q_j$$
        
    \textit{Where $q_i$ is the generalized coordinates and $Q_j$ is the generalized forces.}
      
    \section{\textit{Generalized coordinates}}
        
    \textit{According to our specific system that we are investigating, we have two generalized coordinates in polar coordinate representation, $\theta$ and x.}
        
    \textit{$\theta$ represents the angle formed by the spring component with the fictitious normal of the system at any given time and is a function of time (time-dependent).}
        
    \textit{x represents the extension in the spring component with the equilibrium length at any given time and is a function of time (time - dependent).}
            
    \section{\textit{Generalized forces}}
        
    \textit{According to our specific system that we are investigating, we only have one generalized forces (non-conservative force), the frictional force acting on the path of the system in the form of drag or fluid resistance.}
        
    \textit{Our frictional force can be embedded in Lagrange's equations in the form of the $Q_j$ term aided with the utilisation of the Rayleigh dissipation function.}
        
    \textit{As there isn't any other non-conservative force, the net total sum of the Lagrange's equations with the incorporation of the Rayleigh dissipation function would equal zero.}
        
\section{\textit{Air Drag/Fluid Resistance}}
        
    \textit{Air Drag/Fluid Resistance is the force acting opposite to the relative motion of any object moving in any fluid medium. Drag force is proportional to the square of velocity, as we are dealing with relatively high-speeds, which can be inferred from the huge Reynolds's number.}
            
    \textit{Drag forces decrease fluid velocity relative to the solid mass in the fluid's path.}
            
    \textit{The type of Drag in play in this system is that of an underdamped ($\zeta < 1$) oscillator with viscous drag.}
            
    \textit{\textbf{Note}: $\zeta$ here, in the context of Drag forces and resistive forces symbolizes the damping ratio.}
            
    \textit{The general Drag equation is mathematically defined as,}
            
        $$F_D = \frac{1}{2}\rho v^2C_DA$$
           
    \textit{Where $F_D$ is the Air/Fluid resistance between the mass and the fluid, $\rho$ is the density of the fluid, v is the speed of the object relative to the fluid, $C_D$ is velocity decay constant (damping constant) and A is the cross sectional area.}
            
    \textit{The Drag force can be embedded and modelled into the Euler-Lagrange equation by the aid of the Rayleigh dissipation function and can be generally modeled using Stroke's Law, as we are using object masses that are spherical.}
            
\section{\textit{Rayleigh dissipation function}}
            
    \textit{Secondly, we must also know that,}
            
    \textit{If the frictional force on a particle with velocity $\vec{v}$ can be written as $\vec{F_f} = -\vec{k}\cdot\vec{v}$, the Rayleigh dissipation function can be defined for a system of \textbf{n} particles as,}
            
        $$R/D = \frac{1}{2} \sum_{i=0}^n C_Dv^2_i = \frac{1}{2}C_D \sum_{i=0}^n v^2_i$$
            
    \textit{Where R or D is the Rayleigh dissipation function which is a function used to handle and model the effects of velocity-proportional frictional forces in Lagrangian mechanics, where $C_D$ is the velocity decay constant (damping constant) and $\sum_{i=0}^n v^2_i$ is the sum of all velocities squared in all degrees of freedom pertaining to a mechanical system with i being the number of degrees of freedom in the system.}
            
\section{\textit{Stroke's Law}}\label{slaw}
            
    \textit{We must know the Stroke's Law and the relation it defines to accurately model the change in Air Drag/Fluid Resistance with time.}
            
    \textit{Stroke's Law states a relation for the frictional force (drag force) exerted on spherical objects with very small Reynolds numbers in a viscous fluid. Mathematically,}
            
        $$F_D = 6\pi\mu Rv_D$$
            
    \textit{\textbf{Note}: This relation is only applicable in this scenario because the mass that we are dealing with is spherical.}
            
    \textit{Where $F_D$ is the Air/Fluid resistance between the mass and the fluid, $\mu$ is the dynamic viscosity of the fluid, R is the radius of the mass and $v_D$ is the flow velocity relative to the mass.}
            
\section{\textit{Logarithmic Decrement}}
            
    \textit{The system exhibits an interesting feature, that of constant logarithmic decrements, that is,}
            
        $$\ln{\frac{x_1}{x_2}} = \ln{\frac{x_2}{x_3}} = \ln{\frac{x_3}{x_4}} = \cdots\cdots\cdots$$
            
    \textit{Where $x_n$ and $x_{n + 1}$ are the amplitudes of any two successive peaks ($n \in \mathbb{R}$).}
            
    \textit{Also interestingly that for any two successive peaks, if we define,}
            
        $$\delta = \ln{\left(\frac{x_n}{x_{n + 1}}\right)}$$
            
    \textit{Then,}
            
        $$\zeta = \frac{\delta}{\sqrt{\delta^2 + \left(2\pi\right)^2}}$$
            
    \textit{With the above findings and definitions, we can create an expression for percentage overshoot, that is,}
            
        $$\textit{Percentage Overshoot} = 100\cdot\exp{\left(-\frac{\zeta\pi}{\sqrt{1 - \zeta^2}}\right)}$$
            
    \textit{Consequently we can find an expression for $\zeta$ in terms of \textbf{percentage overshoot} (PO). That is,}
            
        $$\zeta = \frac{-\ln{\left(\frac{PO}{100}\right)}}{\sqrt{\pi^2 + \ln^2{\left(\frac{PO}{100}\right)}}}$$
            
    \textit{\textbf{Note}: $\zeta$ here, in the context of Drag forces and resistive forces symbolizes the damping ratio.}
            
\section{\textit{Computer Simulation Software}}
        
    \textit{To model the chaotic system of the Damped Harmonic Elastic Pendulum we would need the aid of \textbf{computational technology}. Software's such MATLAB, Mathematica or some simple, eccentric computer programming language code in a script format, that would compute and yield solutions and necessary simulations that are required.}
    
    \textit{For the purpose of this investigation we shall be using a \textbf{Python script}, the source code of which can be found in \textbf{appendix \ref{pycode}}, to model and simulate the dynamic and chaotic system.}
        
        


